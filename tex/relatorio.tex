\documentclass[a4paper]{report}

\usepackage[utf8]{inputenc} 
\usepackage[portuges]{babel}
\usepackage{a4wide}
\usepackage{graphicx}

\title{Projeto de Laboratórios de Informática 1\\Grupo 49}
\author{Rui Nuno Vilaça Ribeiro 80207 \and Nelson Tiago Silva Sousa 82053}
\date{\today}

\begin{document}

\maketitle

\begin{abstract}
  Este trabalho sintetiza o nosso projeto de Laboratórios de Informática do curso de Engenharia Informática da Universidade do Minho, que consiste em realizar o jogo \emph{Bomberman} na linguagem Haskell.
  Consideramos que fizemos um bom trabalho, visto que esta foi a primeira vez a realizar um projeto em Haskell.
\end{abstract}

\tableofcontents

\section{Introdução}
\label{intro} 

 Na 1º e 2º fases do projeto de LI1, o nosso maior objetivo foi tentar acabar todas as tarefas sem dar nenhum erro e, ao mesmo tempo, fazer as funções o mais otimizadas possível. Com o passar do tempo fomo-nos deparando com alguns problemas nas seis tarefas que havia para realizar. 
 Os maiores obstáculos que encontramos foram a implementação das \emph{flames} e a alta dependência dos testes do \emph{svn}, ou seja, como esta última foi desenvolvida pelo método de tentativa/erro, levou-nos bastante tempo a acabar. 
 Em seguida, vamos descrever estes problemas mais detalhadamente. 

\section{Descrição do Problema}
\label{sec:problema}

\subsection{Tarefa1}

O problema com que nos deparamos nesta parte foi gerir mapas através de funções (recursivas). Esta tarefa foi também o primeiro contacto que tivemos com o projeto.

\subsection{Tarefa2}

O problema com que nos deparamos nesta parte foi, dada uma descrição do estado do jogo e um comando de um dos jogadores, determinar o efeito desse comando no estado do jogo.

\subsection{Tarefa3}

Nesta parte o problema com que nos deparamos foi, dada uma descrição do estado do jogo, implementar um mecanismo de compressão/descompressão que permitiria poupar espaço em disco.

\subsection{Tarefa4}

O problema com que nos deparamos nesta parte foi determinar o efeito da passagem de um instante de tempo, dada a descrição do estado do jogo. Esta tarefa não nos levantou muitas dificuldades pelo que explicaremos a nossa soluçao na secção seguinte.

\subsection{Tarefa5}

O problema com que nos deparamos nesta parte foi introduzir a parte gráfica no projeto. Por outras palavras, dada a descrição do estado do jogo tínhamos que traduzir essa informação para um mapa gráfico.
Esta foi, a nosso ver, a tarefa mais difícil de realizar uma vez que tivemos grandes problemas ao realizá-la. 

\subsubsection{Flames}

Como já referimos em \ref{intro}, um dos maiores problemas na realizaçao da 2º fase do projeto foi a implementação das \emph{flames}, uma vez que a nossa estratégia para fazer explodir as bombas na tarefa 4 não permitia traduzir facilmente as explosões em \emph{flames} no jogo porque não tínhamos uma lista de coordenadas para desenhar as \emph{flames} já que estavamos a trabalhar com o mapa e as coordenadas numa só \emph{string}.

\subsection{Tarefa6}

O problema com que nos deparamos nesta parte foi implementar um \emph{bot}, ou seja, pensar numa estratégia de jogo e usar funções que fizessem com que ele se movimentasse no mapa, metesse bombas e que tentasse não morrer para, consequentemente, ganhar o jogo. 

\section{Soluções e Testes}
\label{sec:solucoes}

\subsection{Tarefa1}

Para resolver o problema desta tarefa, decidimos gerir cada mapa, linha por linha, pois achamos que essa era a melhor forma para o fazer. No entanto, não encontrámos dificuldades significativas.
Para testar a nossa solução para esta tarefa, fomos analisando os testes automáticos do \emph{svn} e corrigindo as funções que se encontravam com erros.

\subsection{Tarefa2}

Para resolver o problema desta tarefa, decidimos criar várias funções de verificação para que o jogador não pudesse fazer nada de inválido, como por exemplo, passar por paredes ou tijolos.
Para testar a nossa solução para esta tarefa, criamos exemplos de mapas através da tarefa 1 e íamos analisando os testes automáticos que o \emph{svn} fazia às funções da tarefa 2 associadas aos exemplos criados por nós.

\subsection{Tarefa3}

Para resolver o problema desta tarefa, tentámos analisar \emph{padrões} na criação de mapas, como a disposição das paredes ou tijolos ou espaços vazios seguidos. Depois de termos escrito essas funções de compressão, fizemos o contrário, ou seja, as de descompressão.
Para testar a nossa solução para esta tarefa, além de observarmos os testes automáticos que o \emph{svn} fazia às funções da tarefa 3 associadas aos exemplos já existentes, fomos também vendo a percentagem de compressão que nos era atríbuida e assim íammos otimizando as funções. 

\subsection{Tarefa4}

Para resolver o problema desta tarefa, auxiliamo-nos das funções que criamos na \emph{Tarefa 2} e usámo-las associadas à passagem do tempo.
Para testar a nossa solução para esta tarefa utilizamos os testes automáticos do \emph{svn} associados aos nossos exemplos para terminar esta tarefa com sucesso.

\subsection{Tarefa5}

Para resolver o problema desta tarefa, fizemos funções que substituiam os elementos do mapa por imagens (\emph{bitmaps}) e assim tornar melhor a apresentação do jogo.
Para testar a solução para esta tarefa, a única hipótese foi simplesmente ir escrevendo as funções e ir testando, para vermos se tudo continuava correto ou se havia coisas que se alteravam e que não eram do nosso agrado.

\subsubsection{Flames}

A solução para este grande obstáculo foi separar a lista do mapa e das coordenadas das bombas/jogadores em duas listas, uma com o mapa e outra com as coordenadas. Assim conseguimos traduzir muito mais facilmente as explosões em \emph{flames} pois passamos a ter uma lista de coordenadas para desenhá-las. 

\subsection{Tarefa6}

Para resolver o problema desta tarefa, utilizamos funções que, consoante o estado do jogo, fariam o \emph{bot} deslocar-se para o meio do mapa, destruindo os tijolos e que não o permitissem ficar no raio de bombas que detonassem. 
Para testar a nossa solução para esta tarefa, tivemos que ir analisando os resultados dos \emph{torneios} e deste modo fomos melhorando o nosso \emph{bot} pois só assim foi possível mudar/otimizar a estratégia e corrigir erros das funções.

\section{Conclusões}
\label{sec:conclusao}

Para concluir este relatório, temos a dizer que achamos que fomos bem sucedidos na realização deste projeto pois consideramos que o concretizamos sem erros e de um modo bastante otimizado. Naturalmente que com mais tempo e um pouco mais de conhecimentos acerca desta linguagem, talvez conseguíssemos obter melhores resultados. No entanto, estamos bastante confiantes no resultado final e esperamos concretizar mais projetos com sucesso no futuro.

\end{document}\grid
\grid
